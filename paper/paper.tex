%\documentclass[useAMS,usenatbib]{biom}
\documentclass[useAMS,usenatbib,referee]{example/biom}

%
%
%  Papers submitted to Biometrics should ALWAYS be prepared
%  using the referee option!!!!


%%%%% AUTHORS - PLACE YOUR OWN MACROS HERE %%%%%

\def\bSig\mathbf{\Sigma}
\newcommand{\VS}{V\&S}
\newcommand{\tr}{\mbox{tr}}

%  If you have a landscape table you need to use the rotating package

\usepackage[figuresright]{rotating}
\graphicspath{{example/}{graphics/}}
%% \raggedbottom % To avoid glue in typesetteing, sbs>>

%%%%%%%%%%%%%%%%%%%%%%%%%%%%%%%%%%%%%%%%%%%%%%%%

\setcounter{footnote}{2}

\title[Efficient inference using the approximate Riemannian geometry of stochastic population models]{Efficient inference using the approximate Riemannian geometry of stochastic population models}

\author{B. Calderhead$^{*}$\email{b.calderhead@ucl.ac.uk} \\
	   Centre for Mathematics and Physics in the Life Sciences and Experimental Biology,\\University College London, UK
	   \and 
	   C. S. Gillespie$^{*}$\email{colin.gillespie@newcastle.ac.uk}\\
	   School of Mathematics \& Statistics, Newcastle University,
	   NE1 7RU, UK
	   }

\begin{document}

\date{}

\pagerange{\pageref{firstpage}--\pageref{lastpage}} \pubyear{2012}

\volume{XX}
\artmonth{YYYY}
\doi{10.1111/j.1541-0420.2005.00454.x}

%  This label and the label ``lastpage'' are used by the \pagerange
%  command above to give the page range for the article

\label{firstpage}

%  pub the summary here

\begin{abstract}
\begin{itemize}
\item We present efficient and novel methodology for Bayesian inference over stochastic kinetic models with complex dynamics and large numbers of parameters.

Inference for MJP is difficult: non-linear likelihood, computational
  expensive, proposals expensive, analytically intractable likelihood
\item Data: Partially observed discrete, noisy time-course.
\item Existing work for likelihood evaluations: 
\begin{itemize}
\item Likelihood calculations
\begin{itemize}
\item Exact: Boys et al, slow, not noise
\item Approx SDE - Andy and Darren (with Noise)
\item Approx MC - Andy and Colin (no noise)
\item Approx LNA - Mark (no noise), Kolmo' (no noise, but check)
\end{itemize}
\item New:
\begin{itemize}
\item Approximate proposal mechanism based on the geometry of MC
\begin{itemize}
\item Overcome strong correlation structure which is standard with partially
  observed data
\item Fast proposals based on approximate geometry
\item Accept/reject with approximate model before trying exact model
\end{itemize}
\item Inference is exact
\item Including observational error
\item Time course measurements, unlike Mark's LNA
\end{itemize}
\item Examples:
\begin{itemize}
\item Immigration-death model: exact proposal compared with approximate proposal
  (no noise)
\item Aphid model: 3 blocks * 3 water treats * 3 nitro treats with Poisson error
\item Hidden states (counts), updated using standard method
\end{itemize}
\end{itemize}
\end{itemize}
\end{abstract}

%
%  Please place your key words in alphabetical order, separated
%  by semicolons, with the first letter of the first word capitalized,
%  and a period at the end of the list.
%

\begin{keywords}
Stochastic; Moment-closure.
\end{keywords}

\maketitle

\section{Introduction}
\label{s:intro}

\begin{itemize}
\item Inference for MJP is difficult: non-linear likelihood, computational
  expensive, proposals expensive, analytically intractable likelihood
\item Data: Partially observed discrete, noisy time-course.
\item Existing work for likelihood evaluations: 
\begin{itemize}
\item Likelihood calculations
\begin{itemize}
\item Exact: Boys et al, slow, not noise
\item Approx SDE - Andy and Darren (with Noise)
\item Approx MC - Andy and Colin (no noise)
\item Approx LNA - Mark (no noise), Kolmo' (no noise, but check)
\end{itemize}
\end{itemize}
\item Novel stuff
\end{itemize}
Different from Mark's LNA paper:
\begin{itemize}
\item Analysis of impact of the approximate proposal scheme
\item Using Noise
\item Using delayed acceptance scheme so exact inference
\item Time course data
\item Using real data
\end{itemize}

Importance of stochastic population models for further scientific understanding of natural systems vs deterministic models.  For example in ecology, where we often have to deal with discrete noisy data.  Importance of parametric sensitivity has already been shown for determining vital properties of biological systems such as robustness and also in experimental design where the sensitivity gives information about the identifiability of individual rate constants.

The sensitivity of parameters is also of central importance when inferring unknown parameter values for a mathematical model

Performing exact inference over stochastic kinetic models described by Markov Jump Processes is extremely difficult.  First and foremost, the likelihood is usually analytically intractable for models with levels of complexity suitable for describing natural phenomena.

Approximations of the likelihood surface, such as LNA and moment closure.  Parameter inference - want to explore the likelihood surface - can consider it as a Riemannian manifold.  This manifold is an approximation of the true underlying Riemannian manifold of the analytically intractable likelihood.

non-linear likelihood, computational expensive, proposals expensive, analytically intractable likelihood


\section{Methods}

\subsection{Stochastic Population Models}

\subsection{Moment Closure Scheme}

\subsection{Approximate Riemannian Geometry}

\subsection{Delay Acceptance Scheme}

Key point: little overhead for approximate likelihood


\section{Example}

\subsection{Immigration-death}

\begin{itemize}
\item Comparison between approximate proposal and true proposal
\item Comparison between LNA and MC?
\item Three examples: increase correlation
\end{itemize}

\subsection{Independent time course}

Bimodal example??

\subsection{Aphid simulation study}

\section{Aphid real}


\section{Discussion}

\begin{itemize}
\item Use of geometry vital in tackling real problems
\item Could also use other approximation schemes such as the LNA??
\end{itemize}

\backmatter

%%%%%% include this section if you wish to acknowledge people,
%%%%%% grant support, etc.

\section*{Acknowledgements}

The authors thank Professor A. Sen for some helpful suggestions,
Dr C. R. Rangarajan for a critical reading of the original version of the
paper, and an anonymous referee for very useful comments that improved
the presentation of the paper.\vspace*{-8pt}

%%%%%% include this section only if your manuscript refers to supplementary
%%%%%% materials -- see Instructions for Authors at 
%%%%%% http://www.tibs.org/biometrics

\section*{Supplementary Materials}

Web Appendix 1 referenced in Section~\ref{ss:example} is available
with this paper at the Biometrics website on Wiley Online Library.
\vspace*{-8pt}


\begin{thebibliography}{}
\bibitem[\protect\citeauthoryear{Akash and Tirky}{1988}]{b24} 
Akash, F. J. and Tirky, T. (1988). Proper multivariate conditional
autoregressive models for spatial data analysis. {\it Biometrics} {\bf 196,} 173.

\bibitem[\protect\citeauthoryear{Balaram et~al.}{1985a}]{b2} 
Balaram, C. A., Neeraj, G., Haq, H. J., and Cliff, P. E., (1985a). {\it
NIFT\/} 2nd edition. Boca Raton, Florida: Chapman and Hall.

\bibitem[\protect\citeauthoryear{Commel}{1961}]{b18} 
Commel, J. K. (1961). {\it National Sangget Academia}, 2nd edition.
Hoboken, New Jersey: Wiley.

\bibitem[\protect\citeauthoryear{Das  and Patra}{1986}]{b25} 
Das, B. and Patra, H. M. (1986). Using counts to simultaneously estimate
abundance and detection probabilities in a salamander community. {\it
Herpetologica} {\bf 60,} 468--478.

\bibitem[\protect\citeauthoryear{Emrat}{1974}]{b14} Emrat, T. (1974).
{\it Topics in Stochastic Processes.} New York: Academic  Press.

\bibitem[\protect\citeauthoryear{Emrat}{1985}]{b15} Emrat, T. (1985).
{\it Topics in Stochastic Processes.} New York: Academic  Press.

\bibitem[\protect\citeauthoryear{Goldman et~al.}{1987}]{b9} 
Goldman, M. J. and Ewin, A. (1987). A comparison of smoothing
techniques for CD4 data measured with error in a time-dependent Cox
proportional hazards model. {\it Statistics in Medicine} {\bf 17,} 2061--2077.

\bibitem[\protect\citeauthoryear{Goodsman}{1972}]{b8} 
Goodsman, R. C. (1972). Testing hypotheses in the functional linear
model. {\it Scandinavian Journal of Statistics} {\bf 30,} 241--251.

\bibitem[\protect\citeauthoryear{Gooms}{1972}]{b5} Gooms, R. D.
(1972). Goodsman, R.C. (1972). Testing hypotheses in the functional linear
model. {\it Scandinavian Journal of Statistics} {\bf 31,} 315--323.


\bibitem[\protect\citeauthoryear{Gooms  and Wool}{1970}]{b7} Gooms,
R. D. and Wool, N. J. (1970). A comparison of smoothing
techniques for CD4 data measured with errors in a time-dependent Cox
proportional hazards model. {\it Statistics in Medicine} {\bf 17,} 2091--2099.

\bibitem[\protect\citeauthoryear{Hackel}{1985}]{b10} 
Hackel, P. (1985). {\it Topics in Stochastic Processes.} New York: Academic  Press.

\bibitem[\protect\citeauthoryear{\nobreak Holloman, \nobreak Tinku,  and Gerg}{Holloman et~al.}{1979}]{b11}
Holloman, P. M., Tinku, H. A., and Gerg, I. (1979). A comparison of smoothing
techniques for CD4 data measured with techniques a time-dependent Cox
proportional hazards model. {\it Statistics in Medicine} {\bf 17,} 2105--2111.

\bibitem[\protect\citeauthoryear{Juman}{1986}]{b12} 
Juman, M. (1986). Using counts to simultaneously estimate
abundance and detection probabilities in a salamander community. {\it
Herpetologica} {\bf 61,} 482--495.

\bibitem[\protect\citeauthoryear{Kartik B}{1969}]{b13} 
Kartik, B. V. (1969). Using counts to simultaneously estimate
abundance and detection probabilities in a salamander community. {\it
Herpetologica} {\bf 62,} 511--519.

\bibitem[\protect\citeauthoryear{Lenin}{1932}]{b17} 
Lenin, D. B. (1932). A comparison of smoothing
techniques for CD4 data measured with errors in a time-dependent Cox
proportional hazards model. {\it Statistics in Medicine} {\bf 17,} 2125--2135.

\bibitem[\protect\citeauthoryear{O'Rourke}{1979}]{b4} O'Rourke, D. (1996). Industrial ecology: A 
critical review. {\it International Journal of Environment and Pollution}
{\bf 6,} 389--112.

\bibitem[\protect\citeauthoryear{Oman and \nobreak Raj}{1986}]{b19}
Oman, F. M. and  Raj, E. (1986). {\it Topics in Stochastic Process.} New York: Academic  Press.

\bibitem[\protect\citeauthoryear{Prem et~al.}{1963}]{b20} Prem, G. W., Christ, W., Smap, J.,
Wills, J. A. (1963). {\it Topics in Stochastic Processes.} New York: Academic  Press.

\bibitem[\protect\citeauthoryear{Rahul}{1981}]{b1} Rahul, S. R. (1981).
Industrial ecology: A  critical review. {\it International Journal of
Environment and Pollution} {\bf 6,} 371--375.

\bibitem[\protect\citeauthoryear{Ram  and Shyam}{1972}]{b6} Ram, R. D.
and Shyam, S. (1972). {\it Topics in Stochastic Process.} New York: Academic  Press.

\bibitem[\protect\citeauthoryear{Ravan  and Hari}{1983a}]{b21} Ravan,
M. and Hari, S. (1983). Industrial ecology: A  critical review.
{\it International Journal of Environment and Pollution} {\bf 6,} 389--395.

\bibitem[\protect\citeauthoryear{Ravan  and Hari}{1983b}]{b22}
Ravan, M. and Hari, S. (1984). Industrial ecology: A  critical
review. {\it International Journal of Environment and Pollution} {\bf 7,} 512--519.

\bibitem[\protect\citeauthoryear{Sharma et~al.}{1984}]{b16} Sharma,
F. J. (1984). {\it Topics in Stochastic Process.} New York: Academic Press.

\bibitem[\protect\citeauthoryear{Vim   and Potter}{1988}]{b23} Vim, W.
E. and Potter, H. J. (1988). A  critical
review. {\it International Journal of Environment and Pollution} {\bf 9,} 635--641.

\end{thebibliography}

\label{lastpage}

\end{document}
